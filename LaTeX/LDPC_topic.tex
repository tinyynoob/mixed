\documentclass[conference]{IEEEtran}
\IEEEoverridecommandlockouts
\usepackage{cite}
\usepackage{amsmath,amssymb,amsfonts}
\usepackage{algorithmic}
\usepackage{graphicx}
\usepackage{textcomp}
\usepackage{xcolor}
%\def\BibTeX{{\rm B\kern-.05em{\sc i\kern-.025em b}\kern-.08em
%    T\kern-.1667em\lower.7ex\hbox{E}\kern-.125emX}}

\begin{document}
\title{Simulation of the decoding algorithms of the LDPC code}
\author{\IEEEauthorblockN{1\textsuperscript{st} Given Name Surname}
\IEEEauthorblockA{\textit{dept. name of organization (of Aff.)} \\
\textit{name of organization (of Aff.)}\\
City, Country \\
email address or ORCID}
}

\maketitle

\begin{abstract}
This report gives the explanations of the LDPC code and provides the
simulation results for the decoding algorithms of the LDPC code, 
including the sum-product algorithm, the min-sum algorithm, and some 
improved min-sum algorithms, based on C programming.

\end{abstract}

\section{Introduction}
The error correction codes are widely applied in numerous modern communication systems, 
including cellular systems, Wi-Fi, digital televisions e.t.c.. 
This kind of coding techniques is able to revise some misinformation in communications.

Recently, the low-density parity-check (LDPC) code has defeated the turbo code and 
became a mainly used error correction code in 5G NR.

\section{The concept of LDPC code}

In this report, we specify $\mathbb{F}=\{0,1\}$ with the operators XX as our field.



\end{document}